\chapter{Parametri di Configurazione Implementativi}\label{app:params}

Questo capitolo riporta i valori dei parametri di configurazione utilizzati nell'implementazione dell'approccio descritto nei Capitoli \ref{chap:dataset_generation}, \ref{chap:illuminant_estimation} e \ref{chap:testing}. I valori riportati rappresentano le scelte di default adottate nel codice, ottimizzate empiricamente durante la fase di sviluppo.

\section{Parametri di Generazione Video}\label{app:video_params}

La Tabella \ref{tab:video_params} riassume i parametri utilizzati per la generazione delle sequenze video mediante l'effetto Ken Burns con perturbazioni Perlin Noise (cfr. Sezione \ref{sec:generazione_video}).

\begin{table}[htbp]
\centering
\caption{Parametri di default per la generazione video}
\label{tab:video_params}
\begin{tabular}{llcp{6cm}}
\toprule
\textbf{Parametro} & \textbf{Simbolo} & \textbf{Valore} & \textbf{Descrizione} \\
\midrule
Zoom iniziale/finale & $z_{\text{start}}, z_{\text{end}}$ & 2.0 & Fattore di ingrandimento costante \\
Frame totali & $N_{\text{frames}}$ & 60 & Numero di frame per video \\
Intensità rumore & $\sigma_{\text{noise}}$ & 0.9 & Ampiezza perturbazione Perlin (\%) \\
Velocità rumore & $v_{\text{noise}}$ & 0.2 & Frequenza temporale Perlin \\
Interpolazione pan & -- & linear & Tipo di interpolazione per panning \\
Interpolazione zoom & -- & ease\_in\_out & Tipo di interpolazione per zoom \\
\bottomrule
\end{tabular}
\end{table}

\section{Parametri degli Estimatori di Illuminante}\label{app:estimator_params}

\subsection{Parametri degli Algoritmi Tradizionali}

La Tabella \ref{tab:traditional_params} riporta i parametri specifici per gli algoritmi di stima tradizionali (cfr. Sezione \ref{sec:traditional_methods}).

\begin{table}[htbp]
\centering
\caption{Parametri di default per gli estimatori tradizionali}
\label{tab:traditional_params}
\begin{tabular}{llcp{5.5cm}}
\toprule
\textbf{Algoritmo} & \textbf{Parametro} & \textbf{Valore} & \textbf{Descrizione} \\
\midrule
White-Patch & percentile & 95.0 & Percentile invece di max assoluto \\
Shades-of-Gray & $p$ & 6.0 & Ordine norma Minkowski \\
Gray-Edge & $p$ & 6.0 & Ordine norma Minkowski \\
Gray-Edge & $n$ & 1 & Ordine derivata (1 o 2) \\
Gray-Edge & $\sigma$ & 1.0 & Deviazione std. Gaussian blur \\
\bottomrule
\end{tabular}
\end{table}

\subsection{Parametri per l'Estensione Spaziale}

La Tabella \ref{tab:spatial_params} riassume i parametri per le modalità di stima spaziale grid e sliding window (cfr. Sezione \ref{sec:estensione_locale}).

\begin{table}[htbp]
\centering
\caption{Parametri di default per l'estensione spaziale}
\label{tab:spatial_params}
\begin{tabular}{llcp{5.5cm}}
\toprule
\textbf{Modalità} & \textbf{Parametro} & \textbf{Valore} & \textbf{Descrizione} \\
\midrule
\multicolumn{4}{l}{\textit{Modalità Grid}} \\
\midrule
Grid & patch & 32 & Dimensione patch (pixel) \\
\midrule
\multicolumn{4}{l}{\textit{Modalità Sliding Window}} \\
\midrule
Conv & kernel & mean & Tipo kernel (mean/gaussian) \\
Conv & win & 31 & Dimensione finestra (pixel) \\
Conv & stride & 1 & Passo di sottocampionamento \\
Conv & $\sigma$ & win/6 & Sigma per kernel gaussiano \\
\midrule
\multicolumn{4}{l}{\textit{Parametri Comuni}} \\
\midrule
Entrambe & mask\_threshold & 0.5 & Soglia binarizzazione maschera \\
\bottomrule
\end{tabular}
\end{table}

\section{Parametri di Filtraggio Temporale}\label{app:temporal_params}

La Tabella \ref{tab:temporal_params} elenca i parametri utilizzati per il filtraggio temporale delle mappe di illuminazione (cfr. Sezione \ref{sec:estensione_temporale}).

\begin{table}[htbp]
\centering
\caption{Parametri di default per il filtraggio temporale}
\label{tab:temporal_params}
\begin{tabular}{llcp{5.5cm}}
\toprule
\textbf{Filtro} & \textbf{Parametro} & \textbf{Valore} & \textbf{Descrizione} \\
\midrule
EMA & $\alpha$ & 0.6 & Peso frame corrente in [0,1] \\
\midrule
Gaussian & window & 5 & Dimensione finestra temporale \\
Gaussian & $\sigma_t$ & window/3.0 & Deviazione std. temporale \\
Gaussian & padding & edge & Modalità padding ai bordi \\
\bottomrule
\end{tabular}
\end{table}

\section{Parametri di Benchmark}\label{app:benchmark_params}

La Tabella \ref{tab:benchmark_params} descrive i livelli di densità utilizzati per l'esplorazione dello spazio dei parametri durante il benchmark (cfr. Sezione \ref{sec:pipeline_finale}).

\begin{table}[htbp]
\centering
\caption{Livelli di densità per grid search parametrica}
\label{tab:benchmark_params}
\begin{tabular}{lcp{7cm}}
\toprule
\textbf{Livello} & \textbf{N. Config} & \textbf{Descrizione} \\
\midrule
low & $\sim$10 & Esplorazione grossolana con pochi valori chiave \\
medium & $\sim$50 & Esplorazione bilanciata per test preliminari \\
high & $\sim$200 & Esplorazione fine per ottimizzazione accurata \\
massive & $\sim$1000 & Esplorazione esaustiva per analisi completa \\
\bottomrule
\end{tabular}
\end{table}

\section{Note Implementative}\label{app:implementation_notes}

\subsection{Normalizzazione}
Tutti i vettori di illuminante stimati vengono normalizzati con norma L2 prima del confronto con il ground truth:
\begin{equation}
\hat{\mathbf{e}} = \frac{\mathbf{e}}{||\mathbf{e}||_2 + \epsilon}
\end{equation}
dove $\epsilon = 10^{-8}$ previene divisioni per zero.

\subsection{Formato Maschere}
Le maschere di validità sono gestite come array a 3 canali (H,W,3) in virgola mobile [0,1], con replica del canale monocromatico per uniformità con le immagini RGB. Un pixel è considerato valido se la media sui 3 canali supera \texttt{mask\_threshold}.

\subsection{Gestione Errori}
Per patch o regioni completamente mascherate, la pipeline utilizza un fallback a Gray-World globale o, in alternativa, restituisce un vettore neutro $[1/\sqrt{3}, 1/\sqrt{3}, 1/\sqrt{3}]$.

\subsection{Riproducibilità}
La generazione dei seed per i segmenti video utilizza \texttt{np.random.SeedSequence.spawn()} per garantire riproducibilità deterministica anche in presenza di parallelizzazione. Dato un seed base $s_0$, ogni segmento riceve un seed child unico ma deterministico.
