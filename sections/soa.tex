
\chapter{Stato dell'Arte nella Costanza Cromatica}\label{ch:soa}

La costanza cromatica è un problema fondamentale nella visione computazionale che mira a recuperare i colori reali degli oggetti indipendentemente dall'illuminazione della scena. In questo capitolo esaminiamo l'evoluzione degli algoritmi di bilanciamento automatico del bianco (AWB), partendo dai metodi tradizionali basati su ipotesi statistiche fino agli approcci moderni basati su deep learning, includendo una panoramica dettagliata dei dataset disponibili per la valutazione.

\section{Metodi Tradizionali e loro Formulazioni Matematiche}\label{sec:traditional_methods}

I primi algoritmi di AWB si basano su ipotesi statistiche semplici ma efficaci sulla distribuzione dei colori nelle immagini naturali. Questi metodi rimangono rilevanti per la loro efficienza computazionale e sono spesso utilizzati come baseline o in sistemi embedded con risorse limitate.

\subsection{Gray-World}

L'algoritmo \textbf{Gray-World} \cite{zapryanov_automatic_2012} presume che la riflettanza media di una scena sia neutra (grigia). Matematicamente, stima l'illuminante come:

\begin{equation}
\mathbf{e} = \left( \frac{1}{N} \sum_{x} I_R(x), \frac{1}{N} \sum_{x} I_G(x), \frac{1}{N} \sum_{x} I_B(x) \right)
\label{eq:gray_world}
\end{equation}

dove $I_c(x)$ rappresenta il valore del pixel $x$ nel canale colore $c \in \{R, G, B\}$ e $N$ è il numero totale di pixel. I canali vengono quindi bilanciati normalizzando rispetto al canale verde:

\begin{equation}
\mathbf{k} = \left( \frac{e_G}{e_R}, 1, \frac{e_G}{e_B} \right)
\end{equation}

Questo approccio è veloce ($O(N)$) ma fallisce in scene con dominanza cromatica o distribuzione non uniforme dei colori.

\subsection{White-Patch (Max-RGB)}

Il metodo \textbf{White-Patch} \cite{zapryanov_automatic_2012} assume che il pixel più luminoso nell'immagine corrisponda a una superficie bianca o speculare che riflette direttamente l'illuminante:

\begin{equation}
\mathbf{e} = \left( \max_x I_R(x), \max_x I_G(x), \max_x I_B(x) \right)
\label{eq:white_patch}
\end{equation}

Questo metodo è estremamente efficiente ma sensibile al rumore e agli outlier.\footnote{Nell'implementazione pratica, si utilizza il 95$^\circ$ percentile invece del massimo assoluto per aumentare la robustezza agli outlier e ai pixel affetti da rumore del sensore.} Richiede la presenza di superfici bianche o speculari nella scena per funzionare correttamente.

\subsection{Shades-of-Gray}

\textbf{Shades-of-Gray} \cite{zapryanov_automatic_2012} generalizza sia Gray-World che White-Patch utilizzando la norma di Minkowski di ordine $p$:

\begin{equation}
\mathbf{e} = \left( \left( \frac{1}{N} \sum_{x} I_R(x)^p \right)^{1/p}, \left( \frac{1}{N} \sum_{x} I_G(x)^p \right)^{1/p}, \left( \frac{1}{N} \sum_{x} I_B(x)^p \right)^{1/p} \right)
\label{eq:shades_of_gray}
\end{equation}

dove:
- Per $p = 1$ si riduce a Gray-World
- Per $p \to \infty$ converge a White-Patch
- Valori intermedi (tipicamente $p \in [2, 6]$) offrono un compromesso tra robustezza e accuratezza

\subsection{Gray-Edge}

Il metodo \textbf{Gray-Edge} \cite{van_de_weijer_edge-based_2007} sposta l'attenzione dai valori assoluti dei pixel ai gradienti dell'immagine, assumendo che la media delle differenze di colore sui bordi sia acromatica:

\begin{equation}
\mathbf{e} = \left( \left( \frac{1}{N} \sum_{x} |\nabla^n I_R(x)|^p \right)^{1/p}, \left( \frac{1}{N} \sum_{x} |\nabla^n I_G(x)|^p \right)^{1/p}, \left( \frac{1}{N} \sum_{x} |\nabla^n I_B(x)|^p \right)^{1/p} \right)
\label{eq:gray_edge}
\end{equation}

dove $\nabla^n$ rappresenta la derivata di ordine $n$ (tipicamente $n \in \{1, 2\}$) e $p$ è il parametro della norma di Minkowski. Per $n = 0$ si riduce a Shades-of-Gray. Questo approccio è più robusto alle variazioni di illuminazione locale poiché i gradienti sono meno influenzati dall'intensità assoluta dell'illuminazione.

\subsection{Approcci Probabilistici e Gamut Mapping}

Metodi più sofisticati includono il \textbf{gamut mapping} (Forsyth), che modella l'insieme dei colori osservabili sotto diversi illuminanti, e approcci bayesiani \cite{gehler_bayesian_2008} che incorporano prior statistici sulla distribuzione degli illuminanti e delle riflettanze. Questi metodi offrono migliore accuratezza ma richiedono maggiore complessità computazionale e dataset di training.

\section{Estensioni e Varianti dei Metodi Tradizionali}

\subsection{Metodi con Pre-filtraggio}

Weng \textit{et al.} \cite{weng_novel_2005} hanno introdotto l'algoritmo AGWM (Adaptive Gray World Method) che rimuove i pixel altamente saturi prima di applicare Gray-World:

\begin{equation}
\mathbf{e} = \frac{1}{|\Omega|} \sum_{x \in \Omega} I(x), \quad \Omega = \{x : S(x) < \tau\}
\end{equation}

dove $S(x)$ è la saturazione del pixel e $\tau$ è una soglia adattiva. Questo riduce gli errori in scene con colori dominanti.

\subsection{Metodi basati su Greyness}

Thai \textit{et al.} \cite{thai_fast_2016} hanno proposto un metodo che pesa i pixel in base alla loro "grigiezza" nello spazio YCbCr:

\begin{equation}
w(x) = \exp\left(-\alpha \cdot (|Cb(x)| + |Cr(x)|)\right)
\end{equation}

dove $\alpha$ controlla la selettività del peso. I pixel più vicini al grigio neutro contribuiscono maggiormente alla stima dell'illuminante.

\subsection{Gray Color Points (GCP-AWB)}

L'approccio \textbf{GCP-AWB} \cite{huo_robust_2006} identifica punti grigi locali con minima deviazione nella crominanza:

\begin{equation}
\mathcal{G} = \{x : \sqrt{U(x)^2 + V(x)^2} < \theta\}
\end{equation}

dove $(U, V)$ sono le componenti di crominanza e $\theta$ è una soglia adattiva. L'illuminante viene stimato mediando solo sui punti in $\mathcal{G}$, migliorando la robustezza in scene complesse.

\section{Metodi Tradizionali Moderni}

\subsection{Fast Fourier Color Constancy (FFCC)}

\textbf{Fast Fourier Color Constancy (FFCC)} \cite{barron_fast_2017} rappresenta un approccio innovativo che riformula completamente il problema della stima dell'illuminante. Invece di operare direttamente nello spazio RGB, FFCC trasforma l'immagine nello spazio log-croma, dove i valori dei pixel sono rappresentati come logaritmi dei rapporti tra i canali colore ($u = \log(R/G)$ e $v = \log(B/G)$). In questo spazio bidimensionale, che ha la topologia di un toro per la sua periodicità, il problema di stima dell'illuminante diventa un task di localizzazione spaziale.

L'intuizione chiave è che l'istogramma dei colori nello spazio log-croma contiene informazioni sufficienti per identificare l'illuminante. FFCC sfrutta la convoluzione nel dominio delle frequenze tramite FFT per accelerare drasticamente il processo di matching tra l'istogramma osservato e i pattern appresi durante il training. Questo approccio permette di ottenere prestazioni superiori del 13-20\% rispetto ai metodi precedenti, mentre è 250-3000 volte più veloce, raggiungendo circa 700 fps su dispositivi mobili.

Per le applicazioni video, FFCC incorpora anche un meccanismo di smoothing temporale che riduce il flickering mantenendo coerenza tra frame consecutivi. Il metodo produce una distribuzione posteriore completa sugli illuminanti piuttosto che una singola stima puntuale, permettendo analisi più sofisticate dell'incertezza.

\subsection{Integral FFCC (IFFCC)}

\textbf{Integral FFCC (IFFCC)} \cite{wei_integral_2025} estende il framework di FFCC per gestire scene con illuminazione spazialmente variabile, un problema significativamente più complesso. Mentre FFCC assume un singolo illuminante globale, IFFCC riconosce che le scene reali spesso contengono multiple sorgenti luminose che influenzano diverse regioni dell'immagine.

L'innovazione principale di IFFCC sta nell'utilizzo degli istogrammi integrali nello spazio log-croma. Questa struttura dati permette di calcolare efficientemente l'istogramma di qualsiasi regione rettangolare dell'immagine in tempo costante, indipendentemente dalla dimensione della regione. Combinando questa efficienza computazionale con la convoluzione FFT parallelizzata, IFFCC può produrre mappe dense di illuminazione che variano smoothly nello spazio, mantenendo prestazioni real-time anche per immagini ad alta risoluzione.

Il metodo utilizza tecniche di guided filtering per garantire che le transizioni tra regioni con diversi illuminanti siano naturali e coerenti con i bordi dell'immagine. Questo approccio rende IFFCC particolarmente adatto per scene complesse come ambienti interni con illuminazione mista naturale e artificiale.

\section{Approcci di Machine Learning e Deep Learning}

\subsection{K-Nearest Neighbors White Balance (KNN-WB)}

\textbf{KNN-WB} \cite{afifi_deep_2020} utilizza un approccio non parametrico basato su similarità:

\begin{equation}
\mathbf{e} = \sum_{i=1}^{k} w_i \cdot \mathbf{e}_i, \quad w_i = \frac{\exp(-d_i/\sigma)}{\sum_j \exp(-d_j/\sigma)}
\end{equation}

dove $d_i$ è la distanza tra l'immagine query e la $i$-esima immagine più simile nel dataset, e $\mathbf{e}_i$ è l'illuminante corrispondente.

\subsection{Deep White Balance}

Afifi e Brown hanno introdotto \textbf{Deep WB} \cite{afifi_deep_2020}, un'architettura encoder-decoder che produce direttamente immagini corrette:

\begin{equation}
I_{\text{corrected}} = f_{\theta}(I_{\text{input}}, \text{mode})
\end{equation}

dove $f_{\theta}$ è una rete neurale convoluzionale e "mode" specifica il tipo di correzione (indoor/outdoor). L'architettura utilizza skip connections e produce output multi-scala per preservare i dettagli.

\subsection{Deep WB Blending per Multi-Illuminanti}

\textbf{Deep WB Blending} \cite{afifi_auto_2022} affronta scenari multi-illuminante fondendo immagini con diversi preset WB:

\begin{equation}
I_{\text{out}} = \sum_{i=1}^{n} W_i \odot I_i^{\text{WB}}
\end{equation}

dove $I_i^{\text{WB}}$ sono versioni dell'immagine con diversi bilanciamenti, $W_i$ sono mappe di peso apprese dalla rete, e $\odot$ denota il prodotto elemento per elemento. Le mappe di peso sono vincolate a sommare a 1 per ogni pixel:

\begin{equation}
\sum_{i=1}^{n} W_i(x,y) = 1, \quad \forall (x,y)
\end{equation}

\section{Dataset per la Costanza Cromatica}
%TODO: aggiungere esempi visivi 

I progressi nel campo sono stati resi possibili dai numerosi dataset annotati. La Tabella \ref{tab:datasets} fornisce una panoramica completa dei principali dataset disponibili, mentre la Tabella \ref{tab:algorithms} riassume le caratteristiche degli algoritmi discussi.

\subsection{Dataset a Singolo Illuminante}

I dataset classici come \textbf{Color Checker} (Gehler-Shi) \cite{gehler_bayesian_2008} e \textbf{Cube++} \cite{ershov_cube_2020} contengono scene con un unico illuminante globale, utilizzando una carta Macbeth ColorChecker per il ground truth. Il \textbf{NUS 8-Camera} \cite{cheng_illuminant_2014} aumenta la diversità catturando le stesse scene con otto fotocamere diverse, permettendo studi sulla variabilità inter-camera.

L'\textbf{INTEL-TAU} \cite{laakom_intel-tau_2020}, con oltre 7000 immagini ad alta risoluzione, è attualmente uno dei più grandi dataset disponibili e include annotazioni aggiuntive come color shading e spettri luminosi, rendendolo ideale per l'addestramento di metodi deep learning.

\subsection{Dataset Multi-Illuminante}

Per scenari più complessi con illuminazione mista, sono disponibili dataset specializzati:

- \textbf{Beigpour Multi-Illuminant} \cite{beigpour_multi-illuminant_2013}: fornisce annotazioni per-pixel di riflettanza, shading e sorgenti in ambienti controllati
- \textbf{SFU Gray Sphere} e \textbf{Flying Gray Ball} \cite{ciurea_large_2003,aghaei_flying_2020}: utilizzano oggetti di riferimento grigi per stimare illuminanti multipli in scene reali
- \textbf{LSMI} \cite{kim_large_2021}: il dataset più completo per multi-illuminanti, con quasi 7500 immagini e ground truth per-pixel dettagliato

\section{Considerazioni e Scelta del dataset}\label{sec:scelta_dataset}

L'analisi dello stato dell'arte evidenzia un'evoluzione continua dai metodi statistici tradizionali verso approcci data-driven sempre più sofisticati. Mentre i metodi classici mantengono la loro rilevanza per efficienza e interpretabilità, gli approcci basati su deep learning dominano in termini di accuratezza, specialmente in scenari complessi con illuminazione multipla.

\subsection{Il Dataset LSMI in Dettaglio}

Per il presente lavoro di tesi, la scelta del dataset è ricaduta su \textbf{LSMI} (Large Scale Multi-Illuminant) \cite{kim_large_2021} per diverse ragioni fondamentali che lo rendono il più adatto agli obiettivi della ricerca:

\begin{itemize}
\item \textbf{Completezza e dimensione}: Con quasi 7500 immagini, LSMI è attualmente il dataset multi-illuminante più ampio disponibile pubblicamente, fornendo una base statistica robusta per l'addestramento e la valutazione degli algoritmi.

\item \textbf{Ground truth denso per-pixel}: A differenza di altri dataset che forniscono solo annotazioni globali o sparse, LSMI offre mappe di illuminazione complete per ogni pixel. Questa caratteristica è cruciale per il nostro approccio, in quanto permette di generare sequenze video artificiali mantenendo ground truth densi e accurati per ogni frame.

\item \textbf{Diversità delle sorgenti}: Le immagini sono state acquisite con tre dispositivi diversi (Samsung Note20 Ultra, Sony $\alpha$9, Nikon D810), garantendo variabilità nelle caratteristiche dei sensori e robustezza cross-camera degli algoritmi sviluppati.

\item \textbf{Mappe di miscelazione degli illuminanti}: Il dataset include informazioni dettagliate su come multiple sorgenti luminose si combinano in ogni punto dell'immagine, permettendo una modellazione realistica delle transizioni di illuminazione nelle sequenze video generate.
\end{itemize}

La disponibilità di ground truth densi è particolarmente critica per il nostro obiettivo di estendere i metodi di stima dell'illuminante globale a scenari locali e temporalmente coerenti. La possibilità di creare video artificiali con annotazioni complete per ogni pixel e frame rappresenta un vantaggio unico che nessun altro dataset attualmente offre, rendendo LSMI la scelta naturale per questo lavoro di ricerca.

Dal punto di vista tecnico, il dataset è composto da 7491 immagini acquisite in 307 scene diverse, ciascuna contenente da 2 a 5 illuminanti distinti. Il processo di acquisizione prevede il posizionamento di multiple carte ColorChecker in regioni della scena caratterizzate da diversa predominanza di illuminazione, permettendo di ottenere riferimenti ground truth densi mediante tecniche di interpolazione guidata. Le immagini hanno risoluzione variabile (tipicamente tra $3000 \times 4000$ e $4000 \times 6000$ pixel) e coprono una vasta gamma di scenari indoor e outdoor con diverse combinazioni di luce naturale, artificiale e mista. La metodologia di annotazione, basata su segmentazione semi-automatica e validazione manuale delle regioni di influenza degli illuminanti, garantisce ground truth affidabili anche in zone di transizione complesse tra diverse sorgenti luminose.

\clearpage
% Tabella 3.1a - Caratteristiche Principali
\begin{table}[htbp]
\centering
\caption{Caratteristiche principali degli algoritmi di stima dell'illuminante}
\label{tab:algorithms}
\footnotesize
\begin{tabularx}{\textwidth}{lccccXl}
\toprule
\textbf{Algoritmo} & \textbf{Tipo} & \textbf{Illum.} & \textbf{Categ.} & \textbf{Space} & \textbf{Perf.} & \textbf{Ref.} \\
\midrule
\multicolumn{7}{l}{\textit{Metodi Tradizionali}} \\
\midrule
Gray World & img & glob. & trad. & RGB & media & \cite{zapryanov_automatic_2012} \\
AGWM & img & glob. & trad. & RGB & media & \cite{weng_novel_2005} \\
White Patch & img & glob. & trad. & RGB & buona* & \cite{zapryanov_automatic_2012} \\
Gray Edge & img & glob. & trad. & RGB & buona & \cite{van_de_weijer_edge-based_2007} \\
Shades of Gray & img & glob. & trad. & RGB & buona & \cite{zapryanov_automatic_2012} \\
Greyness WA & img & glob. & trad. & YCbCr & alta & \cite{thai_fast_2016} \\
GCP-AWB & img & local & trad. & YUV & buona & \cite{huo_robust_2006} \\
\midrule
\multicolumn{7}{l}{\textit{Metodi Avanzati e Machine Learning}} \\
\midrule
FFCC & img & glob. & ibr. & log-chr. & altiss. & \cite{barron_fast_2017} \\
IFFCC & video & spat.v. & ibr. & log-chr. & altiss. & \cite{wei_integral_2025} \\
KNN-WB & img & glob. & ML & sRGB & media & \cite{afifi_deep_2020} \\
Deep WB & img & glob. & DL & sRGB & alta & \cite{afifi_deep_2020} \\
Deep WB Blend. & img+vid & spat.v. & DL & sRGB & alta & \cite{afifi_auto_2022} \\
\bottomrule
\end{tabularx}
\end{table}

% Tabella 3.1b - Dettagli Implementativi
\begin{table}[htbp]
\centering
\caption{Dettagli implementativi degli algoritmi (continua dalla Tabella \ref{tab:algorithms})}
\footnotesize
\begin{tabularx}{\textwidth}{l>{\raggedright\arraybackslash}Xl>{\raggedright\arraybackslash}X>{\raggedright\arraybackslash}X}
\toprule
\textbf{Algoritmo} & \textbf{Assunzioni Principali} & \textbf{Compl.} & \textbf{Input Richiesti} & \textbf{Estensioni} \\
\midrule
\multicolumn{5}{l}{\textit{Metodi Tradizionali}} \\
\midrule
Gray World & Colore medio grigio & bassa & nessuno & Gray Edge et al. \\
AGWM & Esclude pixel saturi & bassa & nessuno & versioni local \\
White Patch & Pixel più luminoso bianco & bassa & nessuno & filtri anti-rumore \\
Gray Edge & Media gradienti grigia & media & derivate & ordini/pesi \\
Shades of Gray & Estens. GW con $L^p$ & media & parametro $p$ & tuning $p$ \\
Greyness WA & Pixel grigi pesati Cb+Cr & medio-bassa & nessuno & downsampling \\
GCP-AWB & Punti grigi dev. min. U,V & bassa & nessuno & soglia adatt. \\
\midrule
\multicolumn{5}{l}{\textit{Metodi Avanzati e Machine Learning}} \\
\midrule
FFCC & Stima torus log-chroma FFT & molto bassa & nessuno & smooth. temp. \\
IFFCC & Log-chroma illum. locale & molto bassa & nessuno & guided filt. \\
KNN-WB & Inferisce WB da simili KNN & media & dataset sRGB & accelerabile \\
Deep WB & Encoder-decoder DNN AWB & medio-alta & nessuno & edit in/out \\
Deep WB Blend. & Fonde img diversi WB DNN & alta & preset WB & preset var. \\
\bottomrule
\end{tabularx}
\end{table}


\vspace{1.5cm}

% Tabella 3.2a - Informazioni Generali
\begin{table}[htbp]
\centering
\caption{Informazioni generali sui dataset per la stima dell'illuminante}
\label{tab:datasets}
\footnotesize
\begin{tabularx}{\textwidth}{lccXccl}
\toprule
\textbf{Nome} & \textbf{Scena} & \textbf{Illum.} & \textbf{Ground Truth} & \textbf{N. Img} & \textbf{Disp.} & \textbf{Ref.} \\
\midrule
\multicolumn{7}{l}{\textit{Dataset Single-Illuminant}} \\
\midrule
Color Checker & in+out & single & Macbeth ColorChecker & 568 & pubb. & \cite{gehler_bayesian_2008} \\
Cube++ & in+out & single & Macbeth ColorChecker & 1365 & pubb. & \cite{ershov_cube_2020} \\
INTEL-TAU & in+out & single & RAW + ColorChecker & 7022 & pubb. & \cite{laakom_intel-tau_2020} \\
NUS 8-Camera & in+out & single & Macbeth ColorChecker & 1736 & pubb. & \cite{cheng_illuminant_2014} \\
\midrule
\multicolumn{7}{l}{\textit{Dataset Multi-Illuminant}} \\
\midrule
Beigpour Multi-I. & indoor & multi & per-pixel completo & ~1500 & pubb. & \cite{beigpour_multi-illuminant_2013} \\
SFU Gray Sphere & in+out & multi & gray sphere in frame & 11000 & pubb. & \cite{aghaei_flying_2020} \\
LSMI & indoor & multi & per-pixel mixing & 7486 & pubb. & \cite{kim_large_2021} \\
Flying Gray Ball & indoor & multi & per-pixel sparso & 1300 & pubb. & \cite{ciurea_large_2003} \\
\bottomrule
\end{tabularx}
\end{table}

% Tabella 3.2b - Specifiche Tecniche
\begin{table}[htbp]
\centering
\caption{Specifiche tecniche dei dataset (continua dalla Tabella \ref{tab:datasets})}
\footnotesize
\begin{tabularx}{\textwidth}{ll>{\raggedright\arraybackslash}Xl>{\raggedright\arraybackslash}X}
\toprule
\textbf{Nome} & \textbf{Formato} & \textbf{Camere} & \textbf{Risoluzione} & \textbf{Annotazioni Extra} \\
\midrule
\multicolumn{5}{l}{\textit{Dataset Single-Illuminant}} \\
\midrule
Color Checker & RAW+TIFF & Canon 1D, 5D & 3504$\times$2336 / 4368$\times$2912 & RGB, WB mult., chart \\
Cube++ & RAW+sRGB & Canon EOS 550D & 5184$\times$3456 & bbox, RGB, preview \\
INTEL-TAU & TIFF 16bit & Canon 5DSR, Nikon D810, Sony IMX135 & var. alta ris. & RGB, GDPR, spettro \\
NUS 8-Camera & RAW & 8 camere diverse & variabile & cross-camera GT \\
\midrule
\multicolumn{5}{l}{\textit{Dataset Multi-Illuminant}} \\
\midrule
Beigpour Multi-I. & RAW+sRGB & Nikon D300 + Kinect & 4288$\times$2848 + VGA & refl., shading, depth \\
SFU Gray Sphere & JPEG/PNG & Sony VX-2000 & ~720$\times$480 & dominant+second. RGB \\
LSMI & RAW & Note20 Ultra, $\alpha$9, D810 & variabile & mixing maps, CC \\
Flying Gray Ball & RAW+TIFF & Canon 5D Mk II & 5616$\times$3744 & coord ball, segment. \\
\bottomrule
\end{tabularx}
\end{table}


\subsection{L'Assenza di Dataset Video Multi-Illuminante}

Nonostante la ricchezza di dataset disponibili per immagini statiche, allo stato attuale non esistono dataset pubblici di sequenze video annotate con ground truth denso per illuminazione spatialmente variabile. Questa lacuna rappresenta un ostacolo significativo per lo sviluppo e la valutazione di algoritmi di color constancy applicati a scenari video realistici, dove l'illuminazione può variare sia nello spazio che nel tempo.

Le ragioni di questa assenza sono principalmente di natura pratica. L'annotazione frame-by-frame di sequenze video richiederebbe il posizionamento di color checker visibili e stabili in ogni frame, con la conseguente necessità di processare manualmente centinaia o migliaia di frame per ottenere ground truth affidabili. Il costo temporale ed economico di tale processo risulta proibitivo, specialmente per scenari con illuminazione multipla dove ogni frame richiederebbe la segmentazione e l'annotazione di diverse regioni di influenza. Inoltre, la presenza di movimento di camera, occlusioni dinamiche e variazioni temporali dell'illuminazione introduce ulteriori complessità metodologiche che rendono l'acquisizione diretta estremamente difficoltosa.

Questo gap metodologico motiva l'approccio adottato nel presente lavoro: sfruttando le annotazioni dense per-pixel disponibili in LSMI, è possibile generare sequenze video artificiali che mantengono ground truth accurati per ogni frame, permettendo la valutazione sistematica di metodi di stima locale e temporalmente coerente dell'illuminante senza i vincoli proibitivi dell'acquisizione e annotazione manuale.
