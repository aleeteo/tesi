
\chapter{Stato dell'Arte nella Costanza Cromatica}

\section{Metodi Tradizionali}\label{sec:traditional_methods} % (fold) 

I primi algoritmi tradizionali di AWB si basano su semplici ipotesi statistiche.
L'algoritmo \textbf{Gray-World} presume che la riflettanza media della scena sia neutra (grigia) e bilancia i canali RGB in modo che le loro medie coincidano \cite{zapryanov_automatic_2012}.
Il metodo \textbf{White-Patch} (\textit{max-RGB}) presuppone invece che il pixel più luminoso rappresenti un riferimento bianco, stimando l'illuminante dai valori massimi dei tre canali.
Questi approcci sono veloci ma falliscono in condizioni che violano le ipotesi di base (scene monocromatiche, dominanza cromatica, ecc.).

Un'estensione è \textbf{Shades-of-Gray}, che generalizza Gray-World e White-Patch usando la norma di Minkowski di ordine $p$: per $p=1$ si riduce a Gray-World, per $p\to \infty$ a White-Patch, mentre valori intermedi offrono compromessi in termini di robustezza \cite{zapryanov_automatic_2012}.
Il metodo \textbf{Gray-Edge} sposta l'attenzione dai valori assoluti ai gradienti dell'immagine: l'ipotesi è che le differenze medie sui bordi della scena siano acromatiche \cite{van_de_weijer_edge-based_2007}. 

Sono stati esplorati anche criteri più complessi, come il \textbf{gamut mapping} (Forsyth), che modella l'insieme dei colori possibili sotto un dato illuminante e determina la trasformazione necessaria per riportare i colori osservati nello spazio canonico.
Approcci probabilistici, come quelli bayesiani, hanno ulteriormente migliorato le prestazioni ma a costo di una maggiore complessità computazionale \cite{gehler_bayesian_2008}.
% section Metodi Tradizionali (end)

\section{Estensioni dei Metodi Tradizionali}

Sono state proposte diverse varianti per rendere gli approcci classici più robusti. Weng \textit{et al.} hanno introdotto un filtro che rimuove i pixel altamente saturi prima di applicare Gray-World, riducendo gli errori in scene con colori dominanti \cite{weng_novel_2005}. Thai \textit{et al.} hanno invece proposto un metodo basato sulla ``grigiezza'' dei pixel in YCbCr, pesando maggiormente quelli vicini al grigio neutro \cite{thai_fast_2016}. 

Un approccio locale è \textbf{GCP-AWB} (Gray Color Points), che ricerca aree neutre con deviazione minima nella crominanza U-V e calcola da esse l'illuminante \cite{huo_robust_2006}. Questi metodi mostrano come estensioni relativamente semplici possano migliorare robustezza e adattabilità in scenari complessi.

\section{Metodi Moderni Non-Deep}

Algoritmi più recenti hanno riformulato il problema nel dominio delle frequenze. \textbf{Fast Fourier Color Constancy (FFCC)} rappresenta l'illuminante nello spazio log-croma e lo stima attraverso convoluzione veloce, ottenendo alta accuratezza e velocità \cite{barron_fast_2017}. Inoltre, FFCC introduce un meccanismo di filtraggio temporale che lo rende adatto alle sequenze video, riducendo il flickering fra fotogrammi.

La sua estensione, \textbf{Integral FFCC (IFFCC)}, affronta scenari \textit{spazialmente variabili}, producendo mappe di illuminazione locali tramite istogrammi integrali pur mantenendo l'elaborazione in tempo reale \cite{wei_integral_2025}. Questo rende IFFCC particolarmente adatto a scene complesse e pipeline video.

\section{Approcci di Machine Learning e Deep Learning}

Con l'avvento di dataset calibrati e reti neurali, i metodi basati sull'apprendimento hanno raggiunto prestazioni allo stato dell'arte. Un esempio è \textbf{KNN-WB}, che sfrutta ampie collezioni di immagini correttamente bilanciate per stimare l'illuminante tramite k-nearest neighbors \cite{afifi_deep_2020}. 

Afifi e Brown hanno introdotto \textbf{Deep WB}, un'architettura encoder-decoder che, data un'immagine sRGB, produce versioni corrette per illuminanti indoor e outdoor, consentendo correzione e editing flessibili \cite{afifi_deep_2020}. Successivamente, \textbf{Deep WB Blending} ha esteso l'approccio a scenari multi-illuminante, fondendo immagini con diversi preset di WB attraverso mappe di peso apprese localmente \cite{afifi_auto_2022}.

Questi metodi mostrano come le reti neurali possano superare le rigide ipotesi dei metodi classici, gestendo sia casi a singolo illuminante che a illuminante misto.

\section{Dataset per la Costanza Cromatica}

I progressi in questo campo sono stati resi possibili dai dataset annotati, che hanno supportato sia lo sviluppo di nuovi algoritmi sia la loro valutazione comparativa. I dataset differiscono per tipo di scena, numero di immagini, caratteristiche di illuminazione e granularità della ground truth.

Il \textbf{Color Checker} di Gehler-Shi \cite{gehler_bayesian_2008} e \textbf{Cube++} \cite{ershov_cube_2020} contengono scene indoor e outdoor con un unico illuminante noto e forniscono ground truth tramite una carta Macbeth ColorChecker posizionata nelle scene. Questi dataset sono stati a lungo un riferimento standard per il benchmarking di metodi tradizionali e ML.

Il \textbf{NUS 8-Camera} \cite{cheng_illuminant_2014} aumenta la diversità catturando le stesse scene con otto fotocamere diverse, permettendo di studiare la variabilità inter-camera e la robustezza degli algoritmi.

L'\textbf{INTEL-TAU} \cite{laakom_intel-tau_2020} include oltre 7000 immagini ad alta risoluzione acquisite con diverse fotocamere e arricchite con annotazioni aggiuntive, come color shading e spettri luminosi. La sua dimensione e ricchezza di annotazioni lo rendono un benchmark di riferimento per i metodi basati su apprendimento.

Per scenari multi-illuminante, i dataset includono \textbf{Beigpour Multi-Illuminant} \cite{beigpour_multi-illuminant_2013}, che fornisce annotazioni per-pixel di riflettanza e sorgenti in ambienti di laboratorio; \textbf{SFU Gray Sphere} e \textbf{Flying Gray Ball} \cite{ciurea_large_2003,aghaei_flying_2020}, che usano oggetti di riferimento grigi per stimare più illuminanti in scene reali; e infine \textbf{LSMI} \cite{kim_large_2021}, oggi il dataset più completo per il problema multi-illuminante. LSMI contiene quasi 7500 immagini con ground truth per-pixel sotto differenti illuminazioni e dispositivi, fornendo mappe dettagliate di miscelazione degli illuminanti. Questo dataset è diventato essenziale per l'addestramento e la valutazione di metodi spazialmente variabili e deep learning.

In sintesi, i dataset disponibili coprono un ampio spettro di complessità: da immagini a singolo illuminante con annotazioni globali a dataset multi-illuminante con ground truth densi, offrendo una solida base per far avanzare lo stato dell'arte.

\section{Conclusioni}

Lo stato dell'arte nel bilanciamento del bianco mostra un'evoluzione da metodi semplici e veloci basati su ipotesi statistiche ad approcci guidati dai dati capaci di gestire video e scenari multi-illuminante. La sfida attuale è conciliare accuratezza, robustezza e stabilità temporale con l'efficienza computazionale richiesta dalle applicazioni real-time.

